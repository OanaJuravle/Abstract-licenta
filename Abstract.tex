\documentclass{article}

\newcommand\tab[1][1cm]{\hspace*{#1}}

\begin{document}
	\textbf{Abstract Licenta}\\\\
	\tab Frumusetea tehnologiei este ca mereu este intr-o continua dezvoltare. Animatia 3D este un domeniu care a castigat tot mai multa atentie in ultimii 25-30 ani. In prezent, aceasta este folosita in majoritatea domeniilor, de la educatie, la arhitectura si web-design pana la diferite segmente de publicitate. Cu siguranta putem afirma ca si viitorul animatiei 3D arata la fel de promitator.\\
	\tab O alta tehnologie care creste pe zi ce trece este realitatea augmentata. Unul din domeniile in care are cel mai mare impact este cel medical. Aceasta ofera deja simplificarea operatiilor si reducerea complicatiilor post-operatorii.\\
	\tab Lucrarea de fata are ca scop imbinarea celor doua tehnolgii pentru realizarea unei aplicatii care sa faciliteze recuperarea mobilitatii incheieturii mainii sau a degetelor pentru cineva care a suferit recent un accident sau o operatie. Sub forma unor exercitii, aplicatia vine in ajutorul terapeutilor, care pot seta un program de exercitii care trebuie urmat de catre pacient, precum si urmarirea progresului acestuia. Aplicatia foloseste realitatea augmentata pentru a urmari miscarile efectuate de mana pacient, pe care le transpune apoi in mediul de joc si te trimite celor care urmaresc procesul de reabilitate, acestia avand posibilitatea sa seteze dificultatea exercitiilor sau timpul de exersare. Prin aceasta aplicatie se doreste sa se creasca motivarea si interesul pacientilor in a termina un program de recuperare astfel incat acestia sa revina la activitatile de zi cu zi cat mai curand prin aceste exercitiile care pot fi facute de acasa, intre sedintele de terapie.\\
	\tab Aplicatia are suport pentru Android, urmand ca pe termen lung sa fie implementata si ca o aplicatie web.\\	
\end{document}


